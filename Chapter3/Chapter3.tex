\documentclass[12pt]{article}   % The 12pt increases the font size.
\usepackage{amssymb,amsmath,amsthm}
\usepackage{graphicx}

\newtheorem*{proposition}{Proposition}  % This enables \begin{proposition}
\newtheorem*{theorem}{Theorem}  % This enables \begin{theorem}

\setlength{\oddsidemargin}{0in}
\setlength{\textwidth}{6.5in}    % Decrease the side margins

\setlength{\headheight}{0in}
\setlength{\headsep}{0in}
\setlength{\topmargin}{0in}
\setlength{\textheight}{9in}      % Decrease the top and bottom margins

\pagestyle{empty}                 % Turn off page numbering

\begin{document}

\begin{center}
Chapter 3 Solutions \\
From ``Book of Abstract Algebra'' by Charles C. Pinter \\
\today
\end{center}


\bigskip  % Add extra space

\noindent\textbf{A. Examples of Abelian Groups} \smallskip

\noindent\textbf{1.} \quad Let $G = \mathbb{R}$, and let $\ast$ be the binary operation on $G$ defined by
\[
x \ast y  \;=\; x + y + k \\
\]
for all $x , y \in G$ and where $k$ is a fixed constant.

\begin{proposition}
The set $G$, with the operation $\ast$, is an abelian group.
\end{proposition}
\begin{proof}
To prove the set $G$, with the operation $\ast$, is an abelian group we must show that $\ast$ is commutative, associative and that $G$ has an identity element and inverses.

To prove that $*$ is associative let $x$ and $y$ be arbitrary elements of $G$. Then

\begin{equation*}
x * y = x + y + k
\end{equation*}
and
\[
y * x = y + x + k = x + y + k.
\]
Since the two results are the same, the operation $\ast$ is commutative.

Now let $x$, $y$, and $z$ be arbitrary elements of $G$. Then

\begin{align*}
x * (y * z) &= x * (y + z + k) \\
&= x + (y + z + k) + k \\
&= x + y + z + 2k
\end{align*}

and

\begin{align*}
(x * y) * z &= (x + y + k) * z \\
&= (x + y + k) + z + k \\
&= x + y + z + 2k.
\end{align*}
Since the two results are the same, the operation $*$ is associative.


The identity element for $G$ is $-k$ because
\begin{equation*}
x * -k = x + (-k) + k = x
\end{equation*}
and
\begin{equation*}
-k * x = -k + x + k = x.
\end{equation*}

Finally, we see that the inverse with respect to $*$ is $-x - 2k$ because

\begin{equation*}
x * x^{-1} = x + (-x - 2k) + k = x - x - 2k = -k
\end{equation*}
and
\begin{equation*}
x^{-1} * x = (-x - 2k) + x + k = -x - 2k + x + k = -k
\end{equation*}
Therefore, we conclude that the set $G$, with the operation $\ast$, is an abelian group.
\end{proof}


\noindent\textbf{2.} \quad Let $G = \{x \in \mathbb{R}: x \neq 0\}$, and let $\ast$ be the binary operation on $G$ defined by
\begin{equation}
x * y = \frac{xy}{2}
\end{equation}
for all $x, y \in G$.

\begin{proposition}
The set $G$, with the operation $\ast$, is an abelian group.
\end{proposition}
\begin{proof}
To prove the set $G$, with the operation $\ast$, is an abelian group we must show that $\ast$ is commutative, associative and that $G$ has an identity element and inverses.

To prove that $*$ is associative let $x$ and $y$ be arbitrary elements of $G$. Then
\begin{equation*}
x * y = \frac{xy}{2}
\end{equation*}
and
\begin{equation*}
y * x = \frac{xy}{2} = \frac{yx}{2}.
\end{equation*}
Since the two results are the same, the operation $\ast$ is commutative.

Now let $x$, $y$, and $z$ be arbitrary elements of $G$. Then

\begin{equation*}
x * (y * z) = x * \frac{yz}{2} = \frac{\frac{xyz}{2}}{2} = \frac{xyz}{4}
\end{equation*}
and
\begin{equation*}
(x * y) * z = \frac{xy}{2} * z = \frac{\frac{xyz}{2}}{2} = \frac{xyz}{4}
\end{equation*}
Since the two results are the same, the operation $*$ is associative.

The identity element for $G$ is $2$ because
\[
x * 2 =\frac{x2}{2} = x
\]
and
\[
2 * x = \frac{2x}{2} = x.
\]

Finally we see that $\frac{4}{x}$ is the inverse of the operation $\ast$ because
\[
x * \frac{4}{x} = \frac{\frac{4x}{x}}{2} = \frac{4}{2} = 2
\]
and
\[
\frac{4}{x} * x = \frac{\frac{4x}{x}}{2} = \frac{4}{2}.
\]
Therefore, we conclude that the set $G$, with the operation $\ast$, is an abelian group.
\end{proof}

\noindent\textbf{3.} \quad Let $G = \{x \in \mathbb{R}: x \neq -1\}$, and let $\ast$ be the binary operation on $G$ defined by
\begin{equation}
x * y = x + y +xy
\end{equation}
for all $x, y \in G$.

\begin{proposition}
The set $G$, with the operation $\ast$, is an abelian group.
\end{proposition}
\begin{proof}
To prove the set $G$, with the operation $\ast$, is an abelian group we must show that $\ast$ is commutative, associative and that $G$ has an identity element and inverses.

To prove that $*$ is associative let $x$ and $y$ be arbitrary elements of $G$. Then
\begin{equation*}
x * y = x + y + xy
\end{equation*}
and
\begin{equation*}
y * x = y + x + yx = x + y + xy
\end{equation*}
Since the two results are the same, the operation $\ast$ is commutative.

Now let $x$, $y$, and $z$ be arbitrary elements of $G$. Then
\begin{align*}
x * (y * z) &= x * (y + z + yz) \\
&= x + (y + z + yz) = x + (y + z + yz) + x(y + z +yz) \\
&= x + y + z + xy + xz + yz + xyz
\end{align*}
and
\begin{align*}
(x * y) * z &= (x + y + xy) * z \\
&= (x + y + xy) + z + (x + y + xy)z \\
&= x + y + z + xy + xz + yz + xyz
\end{align*}
Since the two results are the same, the operation $*$ is associative.

The identity element for $G$ is $0$ because
\begin{equation*}
x * 0 = x + 0 + x \cdot 0 = x
\end{equation*}
and
\begin{equation*}
0 * x = 0 + x + 0 \cdot x = x.
\end{equation*}

Finally we see that $-\frac{x}{1+x}$ is the inverse of the operation $\ast$ because
\begin{align*}
x * \left( -\frac{x}{1+x} \right) &= x + \left( - \frac{x}{1+x} \right) + x \cdot  \left( - \frac{x}{1+x} \right) \\
&= \frac{x + x^2 - x + x^2}{1 + x} = 0
\end{align*}
and
\begin{align*}
\left( -\frac{x}{1+x} \right) * x &= \left( -\frac{x}{1+x} \right) + x + \left( -\frac{x^2}{1+x} \right) \\
&= \frac{x + x^2 - x - x^2}{1 + x} = 0
\end{align*}
Therefore, we conclude that the set $G$, with the operation $\ast$, is an abelian group.

\end{proof}

\noindent\textbf{4.} \quad Let $G = \{x \in \mathbb{R}: -1 < x < 1\}$, and let $\ast$ be the binary operation on $G$ defined by
\begin{equation}
x * y = \frac{x + y}{xy + 1}
\end{equation}
for all $x, y \in G$.

\begin{proposition}
The set $G$, with the operation $\ast$, is an abelian group.
\end{proposition}

\begin{proof}
To prove the set $G$, with the operation $\ast$, is an abelian group we must show that $\ast$ is commutative, associative and that $G$ has an identity element and inverses.

To prove that $*$ is associative let $x$ and $y$ be arbitrary elements of $G$. Then
\begin{equation*}
x * y = \frac{x + y}{xy + 1}
\end{equation*}
and
\begin{equation*}
y * x = \frac{y + x}{yx + 1} = \frac{x + y}{xy + 1}
\end{equation*}
Since the two results are the same, the operation $\ast$ is commutative.

Now let $x$, $y$, and $z$ be arbitrary elements of $G$. Then
\begin{align*}
x * (y * z) &= x * \frac{y + z}{yz + 1} \\
&= \frac{x + \frac{y + z}{yz + 1}}{\frac{y + z}{yz + 1} + 1} \\
&= x + y + z + xy + xz + yz + xyz + 1
\end{align*}
and
\begin{align*}
(x * y) * z &= \frac{x + y}{xy + 1} *z \\
&= \frac{\frac{x + y }{xy + 1} + z}{\frac{x + y}{xy + 1}z + 1} \\
&= x + y + z + xy + xz + yz + xyz + 1
\end{align*}
Since the two results are the same, the operation $*$ is associative.

The identity element for $G$ is $0$ because
\begin{equation*}
x * 0 = \frac{x + 0}{x \cdot 0 + 1} = \frac{x}{1} = x
\end{equation*}
and
\begin{equation*}
0 * x = \frac{0 + x}{0 \cdot x + 1} = \frac{x}{1} = x
\end{equation*}

Finally we see that $-x$ is the inverse of the operation $\ast$ because
\begin{equation*}
x * -x = \frac{x - x}{x \cdot -x + 1} = \frac{0}{-x^2 + 1} = 0
\end{equation*}
and
\begin{equation*}
-x * x = \frac{-x + x}{-x \cdot x + 1} = \frac{0}{-x^2 + 1} = 0
\end{equation*}

\end{proof}
Therefore, we conclude that the set $G$, with the operation $\ast$, is an abelian group.
\bigskip

\noindent\textbf{B. Groups on the Set} $\mathbb{R} \times \mathbb{R}$ \smallskip

\noindent\textbf{1.} \quad Let $(a,b)*(c,d) = (ad+bc,bd)$ be a binary operation $*$ on the set $G = \{(x,y) \in \mathbb{R} \times \mathbb{R}: y \neq 0\}$ \\

\noindent\textbf{(a)}
\begin{proposition}
The set $G$, with the operation $\ast$, is a group.
\end{proposition}
\begin{proof}
To prove the set $G$, with the operation $\ast$, is a group we must show that $\ast$ is associative and that $G$ has an identity element and inverses.

To prove that $*$ is associative let $a$, $b$, $c$ and $d$ be arbitrary elements of $G$. Then
\begin{align*}
(a,b)*((c,d)*(e,f)) &= (a,b)*(cf + de, df) = (a \cdot df + b \cdot (cf + de), b \cdot df) \\
&= (adf + bcf + bde, bdf)
\end{align*}
and
\begin{align*}
((a,b)*(c,d))*(e,f) &= (ad + bc, bd) * (e,f) = ((ad + bc) \cdot f + bd \cdot e, bd \cdot f) \\
&= (adf + bcf + bde, bdf)
\end{align*}
Since the two results are the same, the operation $*$ is associative.

The identity element for $G$ is $(0,1)$ because
\begin{equation*}
(a,b) * (0,1) = (a + 0, b) = (a,b)
\end{equation*}
and
\begin{equation*}
(0,1) * (a,b) = (0 + a, b) = (a,b)
\end{equation*}

Finally we see that $(\tfrac{-a}{b^2}, \tfrac{1}{b})$ is the inverse of the operation $\ast$ because
\begin{equation*}
(a,b) * \left(-\frac{a}{b^2}, \frac{1}{b} \right) = \left(a \cdot \frac{1}{b} + b \cdot -\frac{a}{b^2}, b \cdot \frac{1}{b} \right) = \left( \frac{ab}{b^2} - \frac{ab}{b^2}, 1 \right) = (0,1)
\end{equation*}
and
\begin{equation*}
\left( -\frac{a}{b^2}, \frac{1}{b} \right) * (a,b) = \left( -\frac{a}{b^2} \cdot b + \frac{a}{b}, \frac{1}{b} \cdot b \right) = \left( -\frac{ab}{b^2} + \frac{ab}{b^2}, 1 \right) = (0,1)
\end{equation*}
Therefore, we conclude that the set $G$, with the operation $\ast$, is a group.
\end{proof}

\bigskip
\noindent\textbf{(b)}
\begin{proposition}
The set $G$, with the operation $\ast$, is an abelian group.
\end{proposition}
\begin{proof}
To prove the set $G$, with the operation $\ast$, is an abelian group we must show that $\ast$ is commutative, associative and that $G$ has an identity element and inverses. In part (a), we have already shown that the operation $*$ is associative and that $G$ has an identity element and inverses. Therefore we must show that the operation $*$ is commutative.

To prove that $*$ is commutative let $a$ and $b$ be arbitrary elements of $G$. Then
\begin{equation*}
(a,b)*(c,d) = (ad + bc, bd)
\end{equation*}
and
\begin{equation*}
(c,d)*(a,b) = (cb +da,db) = (ad + bc, bd)
\end{equation*}
Since the two results are the same, the operation $\ast$ is commutative.

Therefore, we conclude that the set $G$, with the operation $\ast$, is an abelian group.
\end{proof}

\noindent\textbf{2.} \quad Let $(a,b)*(c,d) = (ac, bc + d)$ be a binary operation $*$ on the set $G = \{(x,y) \in \mathbb{R} \times \mathbb{R}: x \neq 0\}$ \\

\noindent\textbf{(a)}
\begin{proposition}
The set $G$, with the operation $\ast$, is a group.
\end{proposition}
\begin{proof}
To prove the set $G$, with the operation $\ast$, is a group we must show that $\ast$ is associative and that $G$ has an identity element and inverses.

To prove that $*$ is associative let $a$, $b$, $c$ and $d$ be arbitrary elements of $G$. Then
\begin{align*}
(a,b) * ((c,d) * (e,f)) &= (a,b) * (ce, de + f) = (a \cdot ce, b \cdot ce + de + f) \\
&= (ace, bce + de + f)
\end{align*}
and
\begin{align*}
((a,b)*(c,d)) * (e,f) &= (ac, bc + d) * (e,f) = (ac \cdot e, (bc + d) \cdot e + f) \\
&= (ace, bce + de + f)
\end{align*}
Since the two results are the same, the operation $*$ is associative.

The identity element for $G$ is $(1,0)$ because
\begin{equation*}
(a,b) * (1,0) = (a \cdot 1, b \cdot 1 + 0) = (a,b)
\end{equation*}
and
\begin{equation*}
(1,0) * (a,b) = (1 \cdot a, 0 \cdot a + b) = (a,b)
\end{equation*}

Finally we see that $(\tfrac{1}{a}, \tfrac{-b}{a})$ is the inverse of the operation $\ast$ because
\begin{equation*}
(a,b) * \left( \frac{1}{a}, -\frac{b}{a} \right) = \left( a \cdot \frac{1}{a}, \frac{b}{a} + \left( -\frac{b}{a} \right) \right) = (1,0)
\end{equation*}
and
\begin{equation*}
\left( \frac{1}{a}, -\frac{b}{a} \right) * (a,b) = \left(\frac{1}{a} \cdot a, -\frac{b}{a} \cdot a + b \right) = (1, -b + b) = (1,0)
\end{equation*}
Therefore, we conclude that the set $G$, with the operation $\ast$, is a group.
\end{proof}

\bigskip
\noindent\textbf{(b)}
\begin{proposition}
The set $G$, with the operation $\ast$, is not an abelian group.
\end{proposition}
\begin{proof}
To prove that the set $G$, with the operation $\ast$, is not an abelian group we must show that the operation $*$ is not commutative.

Let $a$, $b$, $c$ and $d$ be arbitrary elements of $G$. Then
\begin{equation}
(a,b) * (c,d) = (ac,bc+d)
\end{equation}
and
\begin{equation}
(c,d) * (a,b) = (ca, da+b)
\end{equation}
Since the two results are not the same, the operation $\ast$ is not commutative.

Therefore, we conclude that the set $G$, with the operation $\ast$, is not an abelian group.
\end{proof}

\noindent\textbf{3.} \quad Let $(a,b)*(c,d) = (ac, bc + d)$ be a binary operation $*$ on the set $G = \{(x,y) \in \mathbb{R} \times \mathbb{R}\}$ \\

\noindent\textbf{(a)}
\begin{proposition}
The set $G$, with the operation $\ast$, is not a group.
\end{proposition}
\begin{proof}
For the set $G$ with the operation $*$ to be a group, all elements in $G$ must have an inverse with respect to the operation $*$. We have shown in problem 2 that the inverse of the operation $*$ defined by $(a,b)*(c,d) = (ac, bc + d)$ is $(\tfrac{1}{a}, \tfrac{-b}{a})$. If we let $(x,y) = (0,y)$ then we see that the inverse is not defined and it follows that not every element in $G$ has an inverse with respect to the operation $*$. Therefore, the set $G$, with the operation $\ast$, is not a group.
\end{proof}

\bigskip
\noindent\textbf{(b)}
\begin{proposition}
The set $G$, with the operation $\ast$, is not an abelian group.
\end{proposition}
\begin{proof}
To be an abelian group, the set $G$ with the operation $*$ must be a group and the operation $*$ must be commutative. We have shown in part (a) that the set $G$ with the operation $*$ is not a group. Therefore, the set $G$ with the operation $*$ is also not an abelian group. 
\end{proof}
\bigskip

\noindent\textbf{4.} \quad Let $(a,b)*(c,d) = (ac - bd, ad + bc)$ be a binary operation $*$ on the set $G = \{(x,y) \in \mathbb{R} \times \mathbb{R}\}$ with the origin deleted. \\

\noindent\textbf{(a)}
\begin{proposition}
The set $G$, with the operation $\ast$, is a group.
\end{proposition}
\begin{proof}
To prove the set $G$, with the operation $\ast$, is a group we must show that $\ast$ is associative and that $G$ has an identity element and inverses.

To prove that $*$ is associative let $a$, $b$, $c$ and $d$ be arbitrary elements of $G$. Then
\begin{align*}
(a,b) * ((c,d) * (e,f)) &= (a,b) * (ce - df, cf + de) \\
&= (a \cdot (ce - df) - b \cdot (cf + de), a \cdot (cf + de) + (ce - df) \cdot b) \\
&= (ace - adf - bcf - bde, acf + ade + bce - bdf)
\end{align*}
and
\begin{align*}
((a,b) * (c,d)) * (e,f) &= (ac - bd, ad + bc) * (e,f) \\
&= ((ac - bd) \cdot e - (ad + bc) \cdot f, (ac - bd) \cdot f + (ad + bc) \cdot e) \\
&= (ace - adf - bcf - bde, acf + ade + bce - bdf)
\end{align*}

The identity element for $G$ is $(1,0)$ because
\begin{equation*}
(a,b) * (1,0) = (a - 0, 0 + b) = (a,b)
\end{equation*}
and
\begin{equation*}
(1,0) * (a,b) = (a - 0, b + 0) = (a,b)
\end{equation*}

Finally we see that $(\tfrac{a}{a^2 + b^2}, \tfrac{-b}{a^2 + b^2})$ is the inverse of the operation $\ast$ because
\begin{align*}
(a,b) * \left( \frac{a}{a^2 + b^2}, -\frac{b}{a^2 + b^2} \right) &= \left( a \cdot \frac{a}{a^2 + b^2} - b \cdot \frac{-b}{a^2 + b^2}, a \cdot \frac{-b}{a^2 + b^2} + b \cdot \frac{a}{a^2 + b^2} \right) \\
&= (1,0)
\end{align*}
and
\begin{align*}
 \left( \frac{a}{a^2 + b^2}, -\frac{b}{a^2 + b^2} \right) * (a,b) &= \left(\frac{a}{a^2 + b^2} \cdot a - \left(-\frac{b}{a^2 + b^2} \cdot b \right), \frac{a}{a^2 + b^2} \cdot b + \left( -\frac{b}{a^2 + b^2} \cdot a \right) \right) \\
&= (1,0)
\end{align*}
Therefore, we conclude that the set $G$, with the operation $\ast$, is a group.
\end{proof}

\bigskip
\noindent\textbf{(b)}
\begin{proposition}
The set $G$, with the operation $\ast$, is an abelian group.
\end{proposition}
\begin{proof}
To prove the set $G$, with the operation $\ast$, is an abelian group we must show that $\ast$ is commutative, associative and that $G$ has an identity element and inverses. In part (a), we have already shown that the operation $*$ is associative and that $G$ has an identity element and inverses. Therefore we must show that the operation $*$ is commutative.

To prove that $*$ is commutative let $a$, $b$, $c$, and $d$ be arbitrary elements of $G$. Then
\begin{equation*}
(a,b)*(c,d) = (ac - bd, ad + bc)
\end{equation*}
and
\begin{equation*}
(c,d)*(a,b) = (ca - db, cb + da) = (ac - bd, ad + bc)
\end{equation*}
Since the two results are the same, the operation $\ast$ is commutative.

Therefore, we conclude that the set $G$, with the operation $\ast$, is an abelian group.
\end{proof}

\noindent\textbf{5.} \quad No, the operation $*$ in exercise 4, defined as $(a,b)*(c,d) = (ac - bd, ad + bc)$, on the set $\mathbb{R} \times \mathbb{R}$ is not a group because not every element in $\mathbb{R} \times \mathbb{R}$ has an inverse with respect to the operation $*$. From the solution to exercise 4 we know that the inverse with respect to $*$ is $(\tfrac{a}{a^2 + b^2}, \tfrac{-b}{a^2 + b^2})$ and we see that the inverse for $(0,0)$ is not defined. Therefore, the operation $*$ on the set $\mathbb{R} \times \mathbb{R}$ is not a group.


\end{document}