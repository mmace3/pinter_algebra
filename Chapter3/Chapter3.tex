\documentclass[12pt]{article}   % The 12pt increases the font size.
\usepackage{amssymb,amsmath,amsthm}
\usepackage{graphicx}

\newtheorem*{proposition}{Proposition}  % This enables \begin{proposition}
\newtheorem*{theorem}{Theorem}  % This enables \begin{theorem}

\setlength{\oddsidemargin}{0in}
\setlength{\textwidth}{6.5in}    % Decrease the side margins

\setlength{\headheight}{0in}
\setlength{\headsep}{0in}
\setlength{\topmargin}{0in}
\setlength{\textheight}{9in}      % Decrease the top and bottom margins

\pagestyle{empty}                 % Turn off page numbering

\begin{document}

\begin{center}
Chapter 3 Solutions \\
From ``Book of Abstract Algebra'' by Charles C. Pinter \\
\today
\end{center}


\bigskip  % Add extra space

\noindent\textbf{A. Examples of Abelian Groups} \smallskip

\noindent\textbf{1.} \quad Let $G = \mathbb{R}$, and let $\ast$ be the binary operation on $G$ defined by
\[
x \ast y  \;=\; x + y + k \\
\]
for all $x , y \in G$ and where $k$ is a fixed constant.

\begin{proposition}
The set $G$, with the operation $\ast$, is an abelian group.
\end{proposition}
\begin{proof}
To prove the set $G$, with the operation $\ast$, is an abelian group we must show that $\ast$ is commutative, associative and that $G$ has an identity element and inverses.

To prove that $*$ is associative let $x$ and $y$ be arbitrary elements of $G$. Then

\begin{equation*}
x * y = x + y + k
\end{equation*}
and
\[
y * x = y + x + k = x + y + k.
\]
Since the two results are the same, the operation $\ast$ is commutative.

Now let $x$, $y$, and $z$ be arbitrary elements of $G$. Then

\begin{align*}
x * (y * z) &= x * (y + z + k) \\
&= x + (y + z + k) + k \\
&= x + y + z + 2k
\end{align*}

and

\begin{align*}
(x * y) * z &= (x + y + k) * z \\
&= (x + y + k) + z + k \\
&= x + y + z + 2k.
\end{align*}
Since the two results are the same, the operation $*$ is associative.


The identity element for $G$ is $-k$ because
\begin{equation*}
x * -k = x + (-k) + k = x
\end{equation*}
and
\begin{equation*}
-k * x = -k + x + k = x.
\end{equation*}

Finally, we see that the inverse with respect to $*$ is $-x - 2k$ because

\begin{equation*}
x * x^{-1} = x + (-x - 2k) + k = x - x - 2k = -k
\end{equation*}
and
\begin{equation*}
x^{-1} * x = (-x - 2k) + x + k = -x - 2k + x + k = -k
\end{equation*}
Therefore, we conclude that the set $G$, with the operation $\ast$, is an abelian group.
\end{proof}


\noindent\textbf{2.} \quad Let $G = \{x \in \mathbb{R}: x \neq 0\}$, and let $\ast$ be the binary operation on $G$ defined by
\begin{equation}
x * y = \frac{xy}{2}
\end{equation}
for all $x, y \in G$.

\begin{proposition}
The set $G$, with the operation $\ast$, is an abelian group.
\end{proposition}
\begin{proof}
To prove the set $G$, with the operation $\ast$, is an abelian group we must show that $\ast$ is commutative, associative and that $G$ has an identity element and inverses.

To prove that $*$ is associative let $x$ and $y$ be arbitrary elements of $G$. Then
\begin{equation*}
x * y = \frac{xy}{2}
\end{equation*}
and
\begin{equation*}
y * x = \frac{xy}{2} = \frac{yx}{2}.
\end{equation*}
Since the two results are the same, the operation $\ast$ is commutative.

Now let $x$, $y$, and $z$ be arbitrary elements of $G$. Then

\begin{equation*}
x * (y * z) = x * \frac{yz}{2} = \frac{\frac{xyz}{2}}{2} = \frac{xyz}{4}
\end{equation*}
and
\begin{equation*}
(x * y) * z = \frac{xy}{2} * z = \frac{\frac{xyz}{2}}{2} = \frac{xyz}{4}
\end{equation*}
Since the two results are the same, the operation $*$ is associative.

The identity element for $G$ is $2$ because
\[
x * 2 =\frac{x2}{2} = x
\]
and
\[
2 * x = \frac{2x}{2} = x.
\]

Finally we see that $\frac{4}{x}$ is the inverse of the operation $\ast$ because
\[
x * \frac{4}{x} = \frac{\frac{4x}{x}}{2} = \frac{4}{2} = 2
\]
and
\[
\frac{4}{x} * x = \frac{\frac{4x}{x}}{2} = \frac{4}{2}.
\]
Therefore, we conclude that the set $G$, with the operation $\ast$, is an abelian group.
\end{proof}

\noindent\textbf{3.} \quad Let $G = \{x \in \mathbb{R}: x \neq -1\}$, and let $\ast$ be the binary operation on $G$ defined by
\begin{equation}
x * y = x + y +xy
\end{equation}
for all $x, y \in G$.

\begin{proposition}
The set $G$, with the operation $\ast$, is an abelian group.
\end{proposition}
\begin{proof}
To prove the set $G$, with the operation $\ast$, is an abelian group we must show that $\ast$ is commutative, associative and that $G$ has an identity element and inverses.

To prove that $*$ is associative let $x$ and $y$ be arbitrary elements of $G$. Then
\begin{equation*}
x * y = x + y + xy
\end{equation*}
and
\begin{equation*}
y * x = y + x + yx = x + y + xy
\end{equation*}
Since the two results are the same, the operation $\ast$ is commutative.

Now let $x$, $y$, and $z$ be arbitrary elements of $G$. Then
\begin{align*}
x * (y * z) &= x * (y + z + yz) \\
&= x + (y + z + yz) = x + (y + z + yz) + x(y + z +yz) \\
&= x + y + z + xy + xz + yz + xyz
\end{align*}
and
\begin{align*}
(x * y) * z &= (x + y + xy) * z \\
&= (x + y + xy) + z + (x + y + xy)z \\
&= x + y + z + xy + xz + yz + xyz
\end{align*}
Since the two results are the same, the operation $*$ is associative.

The identity element for $G$ is $0$ because
\begin{equation*}
x * 0 = x + 0 + x \cdot 0 = x
\end{equation*}
and
\begin{equation*}
0 * x = 0 + x + 0 \cdot x = x.
\end{equation*}

Finally we see that $-\frac{x}{1+x}$ is the inverse of the operation $\ast$ because
\begin{align*}
x + \left( -\frac{x}{1+x} \right) &= x + \left( - \frac{x}{1+x} \right) + x \cdot  \left( - \frac{x}{1+x} \right) \\
&= \frac{x + x^2 - x + x^2}{1 + x} = 0
\end{align*}
and
\begin{align*}
\left( -\frac{x}{1+x} \right) + x &= \left( -\frac{x}{1+x} \right) + x + \left( -\frac{x^2}{1+x} \right) \\
&= \frac{x + x^2 - x - x^2}{1 + x} = 0
\end{align*}
Therefore, we conclude that the set $G$, with the operation $\ast$, is an abelian group.

\end{proof}

\end{document}