\documentclass[12pt]{article}

\usepackage[left=2cm,right=2cm,top=1.54cm,bottom=2cm]{geometry}
\usepackage{amsfonts}
\usepackage{amsmath}
\usepackage{amssymb}
\usepackage{mathtools}
\DeclarePairedDelimiter{\abs}{\lvert}{\rvert}


\begin{document}
\title{ Chapter 2 - Operations }

\date{\today}
\maketitle
\begin{flushleft}
From "Book of Abstract Algebra" by Charles C. Pinter

\section*{A. Examples of Operations}

Which of the following rules are operations on the indicated set? ($\mathbb{Z}$ designates the set of integers, $\mathbb{Q}$ the rational numbers, and $\mathbb{R}$ the real numbers.) For each rule which is not an operation, explain why it is not.

\begin{enumerate}



\item $a * b = \sqrt{|ab|}$, on the set $\mathbb{Q}$.
\item $a * b = a~\text{ln}~b$, on the set $\{x \in \mathbb{R}: x > 0\}$.
\item $a * b$ is a root of the equation $x^{2} - a^{2}b^{2} = 0$, on the set $\mathbb{R}$.
\item Subtraction, on the set $\mathbb{Z}$.
\item Subtraction, on the set $\{ n \in \mathbb{Z}:n \geq 0 \}$.
\item $a * b = |a-b|$, on the set $\{ n \in \mathbb{Z}: n \geq 0 \}$.

\end{enumerate}

\textit{Solution} \\
\begin{enumerate}
\item This is not an operation on $\mathbb{Q}$ because $a * b$ is not uniquely defined and $\mathbb{Q}$ is not closed under $*$. If $a$ and $b$ are rational numbers they can be written as $a = \frac{c}{d}$ and $b = \frac{e}{f}$ where $c, d, e, \text{and} f$ are integers, $d \neq 0$, and $f \neq 0$. If we let $c = 2$, $d = 1$, $e = 2$, and $f = 1$ then

$$ \sqrt{ \frac{2}{1} \cdot \frac{2}{1} } = \sqrt{ 2 \cdot 2 } = \sqrt{4}$$  

and since $ \sqrt{4} = \pm 2$ we see that $a * b$ is not uniquely defined. Now let $c = 3$, $d = 1$, $e = 2$, and $f = 1$ then

$$ \sqrt{ \frac{3}{1} \cdot \frac{2}{1} } = \sqrt{6}   $$

and we see that there is no rational number $f$ such that $f \cdot f = 6$, therefore $\sqrt{6}$ is not a rational number. Thus, $\mathbb{Q}$ is not closed under $*$.

\item This is not an operation on the set $\{x \in \mathbb{R}: x > 0\}$ because the set $\{x \in \mathbb{R}: x > 0\}$ is not closed under $*$. For example, if we let $a = 2$ and $b = 1$ then

$$ a~\text{ln}~b = 2~\text{ln}~1 = 0$$

and $ 0 \notin \{x \in \mathbb{R}: x > 0\}$. Therefore the set $\{x \in \mathbb{R}: x > 0\}$ is not closed under $*$ and $*$ is not an operation on the set $\{x \in \mathbb{R}: x > 0\}$.

\item This is not an operation on $\mathbb{R}$ because $a * b$ is not uniquely defined. If we solve for $x$ we see

\begin{align*}
x^{2} - a^{2}b^{2} &= 0 \\
x^{2} &= a^{2}b^{2}
\end{align*}

since $a^{2}$ and $b^{2}$ will always be positive numbers, then $a^{2}b^{2}$ will be positive as well. Then solving for $x$ we have

$$ x = \left( ab, - ab \right) $$

and we see that $a * b$ is not uniquely defined and therefore $*$ is not an operation on $\mathbb{R}$.

\item This is an operation on $\mathbb{Z}$.

\item This is not an operation on $\mathbb{Z}$ because the set $\{ n \in \mathbb{Z}:n \geq 0 \}$ is not closed under $*$. For example, if we let $a = 10$ and $b = 5$ then $5-10=-5$ and $-5 \notin \{ n \in \mathbb{Z}:n \geq 0 \}$. Therefore the set $\{ n \in \mathbb{Z}:n \geq 0 \}$ is not closed under $*$.

\item This is an operation on the set $\{ n \in \mathbb{Z}: n \geq 0 \}$.

\end{enumerate}


\section*{B. Properties of Operations}

Each of the following is an operation $*$ on $\mathbb{R}$. Indicate whether or not

\renewcommand{\theenumi}{(\roman{enumi})}
\begin{enumerate}
  \item it is commutative,
  \item it is associative,
  \item $\mathbb{R}$ has and identity element with respect to $*$,
  \item every $x \in \mathbb{R}$ has an inverse with respect to $*$.
\end{enumerate}

\textbf{Instructions} For (i), compute $x * y$ and $y * x$, and verify whether or not they are equal. For (ii), compute $x * (y * z)$ and $(x * y) * z$, and verify whether or not they are equal. For (iii), first solve the equation $x * e = x$ for $e$; if the equation cannot be solved, there is identity element. If it \textit{can} be solved, it is still necessary to check that $e * x = x * e = x$ for any $x \in \mathbb{R}$. If it checks, then $e$ is an identity element. For (iv), first not that it there is no identity element, there can be no inverses. If there is an identity element $e$, first solve the equation $x * x^{'} = e$ for $x^{'}$; if the equation cannot be solved, $x$ does not have an inverse. If it can be solved, check to make sure that $x * x^{'} = x^{'} * x = e$. If this checks, $x^{'}$ is the inverse of $x$.

\renewcommand{\theenumi}{\arabic{enumi}}
\begin{enumerate}

\item $x * y = \sqrt{x^2 + y^2}$
\item $x * y = \abs{ x + y }$
\item $x * y = \abs{ xy }$
\item $x * y = x - y $
\item $x * y = xy + 1 $
\item $x * y = \text{max}\{x,y\}$
\item $x * y = \frac{xy}{x + y + 1}$
\end{enumerate}

\textit{Solution}

\begin{enumerate}

\item
\begin{tabular}{c c c c}
\textit{Commutative} & \textit{Associative} & \textit{Identity} & \textit{Inverses} \\
Yes $\boxtimes$ No $\square$ & Yes $\boxtimes$ No $\square$ & Yes $\boxtimes$ No $\square$ & Yes $\boxtimes$ No $\square$ \\
\end{tabular}
\end{enumerate}
\renewcommand{\theenumi}{(\roman{enumi})}
\begin{enumerate}

\item $x *y = \sqrt{x^2 + y^2}$ \\
 $ y *x = \sqrt{y^2 + x^2} = \sqrt{x^2 + y^2}$ because $+$ is a commutative operation on $\mathbb{R}$ \\
(\textit{Thus $*$ is commutative.})

\item $x * \left( y * z \right) = x *\left(\sqrt{y^2 + z^2} \right) = \sqrt{x^2 + \left(\sqrt{y^2 + z^2} \right)^2} = \sqrt{x^2 + y^2 + z^2}$ \\
$\left( x * y \right) * z = \sqrt( x^2 + y^2 ) * z = \sqrt{\left( \sqrt{x ^2 + y^2} \right)^2 + z^2} = \sqrt{x^2 + y^2 + z^2}$ \\
(\textit{Thus $*$ is associative.})

\item Solve $x*e= x$ for $x$: $\sqrt{x^2 + e^2} = x$ therefore $e = 0$. \\
Check: $x * 0 = \sqrt{x^2 + 0} = \sqrt{x^2} = x$; $0 *x = \sqrt{0 + x^2} = \sqrt{x^2} = x$ \\
Therefore, $0$ is the identity element. \\
(\textit{$*$ has an identity element})

\item Solve $x * x^{'} = e$ for $x^{'}$.

\begin{align*}
x * x^{'} &= e \\
\sqrt{x^2 + {x^{'}}^{2}} &= 0 \\
x^{'} &= \sqrt{-x^2} \\
x^{'} &= \pm ix
\end{align*}

where $i$ is the imaginary unit.

Now check that $x * x^{'} = x^{'} * x = e$. 

If $x^{'} = ix$ then 

\begin{align*}
x * x^{'} &= \sqrt{x^2 + (xi)^2} \\
&= \sqrt{x^2 + (x^2 \cdot i^2)} \\
&= \sqrt{x^2(1+i^2)} \\
&= \sqrt{x^2(1+(-1))} \\
&= \sqrt{x^2 \cdot 0} = 0
\end{align*}

and

\begin{align*}
x^{'} * x &= \sqrt{(xi)^2 + x^2} \\
&= \sqrt{(x^2 \cdot i^2) + x^2} \\
&= \sqrt{(1+i^2)x^2} \\
&= \sqrt{(1+(-1))x^2} \\
&= \sqrt{0 \cdot x^2 } = 0
\end{align*}

If $x^{'} = -ix$ then 

\begin{align*}
x * x^{'} &= \sqrt{x^2 + (-xi)^2} \\
&= \sqrt{x^2 + (xi)^2} ~~~~~~~~  \text{exponent rule:}~(-a)^n = a^n, \text{if} ~n ~\text{is even}\\
&= \sqrt{x^2 + (x^2 \cdot i^2)} \\
&= \sqrt{x^2(1+i^2)} \\
&= \sqrt{x^2(1+(-1))} \\
&= \sqrt{x^2 \cdot 0} = 0
\end{align*}

and

\begin{align*}
x^{'} * x &= \sqrt{(-xi)^2 + x^2} \\
&= \sqrt{(xi)^2 + x^2}  ~~~~~~~~  \text{exponent rule:}~(-a)^n = a^n, \text{if} ~n ~\text{is even}\\
&= \sqrt{(x^2 \cdot i^2) + x^2} \\
&= \sqrt{(1+i^2)x^2} \\
&= \sqrt{(1+(-1))x^2} \\
&= \sqrt{0 \cdot x^2 } = 0
\end{align*}

Therefore, $\pm ix$ is the inverse of x. \\
(\textit{Every element has an inverse})

\end{enumerate}

\renewcommand{\theenumi}{\arabic{enumi}}
\begin{enumerate}\addtocounter{enumi}{1}

\item 
\begin{tabular}{c c c c}
\textit{Commutative} & \textit{Associative} & \textit{Identity} & \textit{Inverses} \\
Yes $\boxtimes$ No $\square$ & Yes $\square$ No $\boxtimes$ & Yes $\boxtimes$ No $\square$ & Yes $\boxtimes$ No $\square$ \\
\end{tabular}
\end{enumerate}

\renewcommand{\theenumi}{(\roman{enumi})}
\begin{enumerate} 
\item $x * y = \abs{x + y}$ \\
$y * x = \abs{y + x} = \abs{x + y}$  because $+$ is a commutative operation on $\mathbb{R}$ \\
(\textit{Thus $*$ is commutative.})

\item 
$x * (y * z) = x * \abs{y + z} = \abs{x + \abs{y + z}}$ \\
Let $x = -5$, $ y = 10$, and $z = -20$, then \\
$\abs{-5 + \abs{10-20}} = \abs{-5 + \abs{-10}} = \abs{-5 + 10} = \abs{5} = 5$ \\
$(x * y) * z = \abs{x + y} * z = \abs{\abs{x + y} + z}$ \\
Again, let $x = -5$, $ y = 10$, and $z = -20$, then \\
$\abs{\abs{-5 + 10}-20} = \abs{\abs{5}-20} = \abs{5-20} = \abs{-15} = 15$ \\
Since $5 \neq 15$ then $*$ is not associative on $\mathbb{R}$

\item Solve $x*e=x$ for $e$: ~~~ $\abs{x + e} = x \Rightarrow e = 0$ \\
Check: $x * 0 = \abs{x + 0} = x$; $0 * x = \abs{0 + x} = \abs{x} = x$. \\
Therefore $0$ is the identity element.

\item Solve $x * x^{'} = 0$ for $x^{'}$: ~~~~ $\abs{x + x^{'}} = 0 \Rightarrow x^{'} = -x$ \\
Check: $x * x^{'} = \abs{x - x} = 0$ \\
$x^{'} * x = \abs{-x + x} = 0$. Therefore, $-x$ is the inverse of $x$.

\end{enumerate}

\renewcommand{\theenumi}{\arabic{enumi}}
\begin{enumerate}\addtocounter{enumi}{2}

\item 
\begin{tabular}{c c c c}
\textit{Commutative} & \textit{Associative} & \textit{Identity} & \textit{Inverses} \\
Yes $\boxtimes$ No $\square$ & Yes $\square$ No $\boxtimes$ & Yes $\boxtimes$ No $\square$ & Yes $\boxtimes$ No $\square$ \\
\end{tabular}
\end{enumerate}









\end{flushleft}




\end{document}
 